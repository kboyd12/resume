\documentclass[
	%a4paper, % Uncomment for A4 paper size (default is US letter)
	11pt, % Default font size, can use 10pt, 11pt or 12pt
]{resume} % Use the resume class

\usepackage{ebgaramond} % Use the EB Garamond font

%------------------------------------------------

\name{Kaleb Boyd} % Your name to appear at the top

% You can use the \address command up to 3 times for 3 different addresses or pieces of contact information
% Any new lines (\\) you use in the \address commands will be converted to symbols, so each address will appear as a single line.

\address{Buford, Georgia} % A secondary address (optional)

\address{(770)~$\cdot$~633~$\cdot$~4258 \\ ktboyd43@gmail.com} % Contact information

%----------------------------------------------------------------------------------------

\begin{document}

%----------------------------------------------------------------------------------------
%	EDUCATION SECTION
%----------------------------------------------------------------------------------------

\begin{rSection}{Education}

	\textbf{Georgia Institute of Technology} \hfill \textit{August 2024} \\
	Master of Science: Analytics \\
	Focus in Computational Data Analytics (Reinforcement Learning, High-Dimensional Data Analysis)

	\textbf{University of Georgia} \hfill \textit{May 2018} \\
	Bachelor of Science: Civil Engineering

\end{rSection}

%----------------------------------------------------------------------------------------
%	WORK EXPERIENCE SECTION
%----------------------------------------------------------------------------------------

\begin{rSection}{Experience}
	%------------------------------------------------
	\begin{rSubsection}{Secured Transportation Services}{October 2024 - Present}{Data Scientist}{Buford, GA}
        \item Developing a custom in-house Large Language Model (LLM) tailored for company needs.
        \item Utilizing local LLMs fine-tuned with Low-Rank Adaptation (LoRA) and In-Context Learning techniques, including Retrieval-Augmented Generation (RAG).
        \item Enhancing company capabilities by leveraging generative AI to improve deliverable speed and accuracy.
        \item Implemented automation engineering solutions, reducing processing time of certain tasks from 5 hours to minutes.

	\begin{rSubsection}{United States Navy}{May 2023 - September 2024}{Cyber Threat Intelligence Analyst}{Pensacola, FL}
		\item Served as a Cyber Threat Intelligence Analyst at Naval Information Operations Command - Pensacola, responsible for implementing defensive and offensive cyberspace mission tasking.
		\item Collected and processed intelligence to generate products satisfying cyberspace intelligence requirements and essential elements of information.
		\item Developed target packages supporting cyberspace operations and disseminated cyber intelligence information through briefings, written reports, and messages.
        \item Deployed to Key Infrastructure and Critical Resource DODIN sites in response to intelligence on potential Advanced Persistent Threat (APT) cyber activity, providing targeted defense and safeguarding these critical networks.
		\item Utilized the HELK stack (Hunting, Elasticsearch, Logstash, Kibana) to identify and remediate potential malicious cyber activity on numerous Department of Defense (DoD) Networks.
	    \item Applied advanced Artificial Intelligence techniques, including Natural Language Processing and Machine Learning, to enhance network traffic anomaly detection and strengthen cybersecurity measures within the Department of Defense. These efforts led to a 90\% reduction in processing time, significantly improving defense against malicious actors in the cyber domain.

	\end{rSubsection}
	%------------------------------------------------

	\begin{rSubsection}{United States Navy}{October 2018 - May 2023}{GEOINT Analyst}{Washington, D.C.}
		\item Attached to Fleet Intelligence Detachment, D.C. (FID DC) as a Geospatial Intelligence Analyst.
		\item Utilized imagery analysis to create enhanced imagery products adhering to National Geospatial Intelligence (NGA) standards through analyzing overhead and aerial imagery developed by photographic and electronic means.
		\item Deployed aboard the USS CARL VINSON CVN-70 (July 2021 - Early 2022) supporting Carrier Strike Group One (CSG-1) and multi-national operations afloat.
		\item Developed data analysis tools using Python that provide a faster and more robust way to filter, organize, and display data collected daily via overhead sensors.
		\item Utilized QGIS software with Python to display geospatial trend analysis of the maritime and air domain for the area of responsibility of CSG-1 resulting in an unprecedented level of awareness that is now being adopted fleet-wide.
		\item Created and maintained an SQL database fusing data from four on-board intelligence watch floors to establish better cohesion and communication of data across the intelligence enterprise.
		\item Assisted in the development of data engineering and data analysis tool sets for the Navy’s Project Overmatch program.
        \item Served as the first tactical afloat data scientist, pioneering the integration of data science on deployment to enhance decision-making and operational productivity. This role catalyzed the 'Data Science at Sea' initiative, now being adopted fleet-wide to drive data-informed strategies across naval operations.
        \item Contributed to the Data Science Working Group for Naval Information Forces, advocating for improved data literacy and specialized training to support tactical data scientists in operational settings.
		\item Served as departmental Leading Petty Officer for Training directly responsible for managing a team of 7 sailors who provided qualification and pre-deployment training to over 100 sailors.
	\end{rSubsection}

	%------------------------------------------------


\end{rSection}

%----------------------------------------------------------------------------------------
%	TECHNICAL STRENGTHS SECTION
%----------------------------------------------------------------------------------------
\begin{rSection}{Certifications}
	\begin{tabular}{@{} >{\bfseries}l @{\hspace{6ex}} l @{}}
		CompTIA & Cybersecurity Analyst (CySA+)                                     \\
		National Geospatial-Intelligence Agency & GEOINT Professional Certification - Fundamental, Imagery Analysis\\
		Department of Defense & Intelligence Fundamentals Professional Certification\\
	\end{tabular}
\end{rSection}


\begin{rSection}{Technical Strengths}

	\begin{tabular}{@{} >{\bfseries}l @{\hspace{6ex}} l @{}}
		Computer Languages            & Python, R, JavaScript, Bash, Rust                                     \\
		Data Analysis / Visualization & Pandas, Numpy, Matplotlib, Plotly, Dash, Seaborn, ggplot, D3          \\
		Data Engineering              & MySQL, PostgreSQL, Redis, Databricks, Elasticsearch, Logstash, Kibana \\
		Machine Learning Frameworks   & Tensorflow, Keras, Scikit-learn, PyTorch                                       \\
		Back-end Web Development      & Django, Flask, FastAPI, Actix-Web, Docker, Docker Compose            \\
        Front-end Web Development     & HTML, CSS, React, Vue \\
	\end{tabular}
\end{rSection}

%----------------------------------------------------------------------------------------
%	EXAMPLE SECTION
%----------------------------------------------------------------------------------------

%\begin{rSection}{Section Name}

%Section content\ldots

%\end{rSection}

%----------------------------------------------------------------------------------------

\end{document}
